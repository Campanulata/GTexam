\question[6] 已知全集$U=\{-1 , 0 , 1 , 2 , 3\}$,集合$A=\{0 , 1 , 2\}$,$B=\{-1 , 0 , 1\}$,则$( \complement  _{U} A) \cap B=$\key{$A$}.
\fourchoices{$\{-1\}$}{$\{0 , 1\}$}{$\{-1 , 2 , 3\}$}{$\{-1 , 0 , 1 , 3\}$}
\begin{solution}{4cm}

\end{solution}


\question[4] 渐进线方程为$x \pm y=0$的双曲线的离心率是\key{$C$}.
\fourchoices{$ \dfrac { \sqrt {2}}{2}$}{$1$}{$ \sqrt {2}$}{$2$}
\begin{solution}{4cm}

\end{solution}



\question[4] 若实数$x$,$y$满足约束条件$\left\{\begin{array}{l}{x-3 y+4 \geqslant 0} \\ {3 x-y-4 \leqslant 0} \\ {x+y \geqslant 0}\end{array}\right.$,则$z=3x+2y$的最大值是\key{$C$}.


\fourchoices{$-1$         }{$1$       }{$10$     }{$12$ }
\begin{solution}{4cm}

\end{solution}



\question[4] 祖暅是我国南北朝时代的伟大科学家,他提出的 " 幂势既同,则积不容异 " 称为祖暅原理,利用该原理可以得到柱体的体积公式$V _{\text{柱体}} =Sh$,其中$S$是柱体的底面积,$h$是柱体的高$.$若某柱体的三视图如图所示$($单位:$cm)$,则该柱体的体积$($单位:$cm ^{3} )$是\key{$B$}.


\fourchoices{$158$}{$162$}{$182$}{$324$}
\begin{center}
\captionof{figure}{第4题}
\vspace{0.5cm}
\end{center}
\begin{solution}{4cm}

\end{solution}



\question[4] 若$a  \gt  0$,$b  \gt  0$,则 " $a+b\leqslant 4$ " 是 " $ab\leqslant 4$ " 的\key{$A$}.
\fourchoices{充分不必要条件}{必要不充分条件}{充分必要条件}{既不充分也不必要条件}
\begin{solution}{4cm}

\end{solution}



\question[4] 在同一直角坐标系中,函数$y= \dfrac {1}{a^{x}}$,$y=1og _{a} (x+ \dfrac {1}{2} )(a  \gt  0$且$a\neq 1)$的图象可能是\key{$D$}.
\fourchoices{见下图.}{见下图.}{见下图.}{见下图.}
\begin{center}
\vspace{0.5cm}
\end{center}
\begin{solution}{4cm}

\end{solution}



\question[4] 设$0  \lt  a  \lt  1.$随机变量$X$的分布列是

\[\begin{array}{|c|c|c|c|}
\hline
X & 0 & a & 1\\
\hline
P & \dfrac {1}{3} & \dfrac {1}{3} & \dfrac {1}{3}\\
\hline
\end{array}\]


则当$a$在$(0 , 1)$内增大时,\key{$D$}.
\fourchoices{$D(X)$增大}{$D(X)$减小}{$D(X)$先增大后减小}{$D(X)$先减小后增大}
\begin{solution}{4cm}

\end{solution}



\question[4] 设三棱锥$V-ABC$的底面是正三角形,侧棱长均相等,$P$是棱$VA$上的点$($不含端点$).$记直线$PB$与直线$AC$所成角为$ \alpha $,直线$PB$与平面$ABC$所成角为$ \beta $,二面角$P-AC-B$的平面角为$γ$,则\key{$B$}.
\fourchoices{$ \beta   \lt  γ$,$ \alpha   \lt  γ$}{$ \beta   \lt   \alpha $,$ \beta   \lt  γ$}{$ \beta   \lt   \alpha $,$γ  \lt   \alpha $}{$ \alpha   \lt   \beta $,$γ  \lt   \beta $}
\begin{solution}{4cm}

\end{solution}



\question[4] 设$a$,$b \in R$,函数$f(x)= \begin{cases} x,x  \lt  0, \\ \dfrac {1}{3}x^{3}- \dfrac {1}{2}(a+1)x^{2}+ax,x\geqslant 0.\end{cases}$若函数$y=f(x)-ax-b$恰有$3$个零点,则\key{$C$}.
\fourchoices{$a  \lt  -1$,$b  \lt  0$}{$a  \lt  -1$,$b  \gt  0$}{$a  \gt  -1$,$b  \lt  0$}{$a  \gt  -1$,$b  \gt  0$}
\begin{solution}{4cm}

\end{solution}



\question[4] 设$a$,$b \in R$,数列$\{a _{n} \}$满足$a _{1} =a$,$a _{n+1} =a _{n} ^{2} +b$,$n \in N ^{*}$,则\key{$A$}.
\fourchoices{当$b= \dfrac {1}{2}$时,$a _{10}  \gt  10$}{当$b= \dfrac {1}{4}$时,$a _{10}  \gt  10$}{当$b=-2$时,$a _{10}  \gt  10$}{当$b=-4$时,$a _{10}  \gt  10$}
\begin{solution}{4cm}

\end{solution}
