\question[4] 复数$z= \dfrac {1}{1+i} (i$为虚数单位$)$,则$|z|=$\key{$ \dfrac { \sqrt {2}}{2}$}.
\begin{solution}{4cm}

\end{solution}



\question[6] 已知圆$C$的圆心坐标是$(0 , m)$,半径长是$r.$若直线$2x-y+3=0$与圆$C$相切于点$A(-2 , -1)$,则$m=$\key{$-2$  $ \sqrt {5}$  },$r=$
\begin{solution}{4cm}

\end{solution}



\question[6] 在二项式$( \sqrt {2} +x) ^{9}$展开式中,常数项和系数为有理数的项的个数分别是\key{$16 \sqrt {2}$ , $5$  }
\begin{solution}{4cm}

\end{solution}



\question[6] 在$\triangle ABC$中,$ \angle ABC=90^{\small \circ}$,$AB=4$,$BC=3$,点$D$在线段$AC$上,若$ \angle BDC=45^{\small \circ}$,则$BD=$\key{$ \dfrac {12 \sqrt {2}}{5}$  $ \dfrac {7 \sqrt {2}}{10}$  },$\cos \angle ABD=$
\begin{solution}{4cm}

\end{solution}



\question[4] 已知椭圆$ \dfrac {x^{2}}{9} + \dfrac {y^{2}}{5} =1$的左焦点为$F$,点$P$在椭圆上且在$x$轴的上方$.$若线段$PF$的中点在以原点$O$为圆心,$|OF|$为半径的圆上,则直线$PF$的斜率是\key{$ \sqrt {15}$}.
\begin{solution}{4cm}

\end{solution}



\question[4] 已知$a \in \mathbf R$,函数$f(x)=ax ^{3} -x.$若存在$t \in \mathbf R$,使得$|f(t+2)-f(t)|\leqslant \dfrac {2}{3}$,则实数$a$的最大值是\key{$ \dfrac {4}{3}$}.
\begin{solution}{4cm}

\end{solution}



\question[6] 已知正方形$ABCD$的边长为$1.$当每个$ \lambda  _{i} (i=1 , 2 , 3 , 4 , 5 , 6)$取遍$ \pm 1$时,$| \lambda  _{1} \overrightarrow{AB} + \lambda  _{2} \overrightarrow{BC} + \lambda  _{3} \overrightarrow{CD} + \lambda  _{4} \overrightarrow{DA} + \lambda  _{5} \overrightarrow{AC} + \lambda  _{6} \overrightarrow{BD} |$的最小值是\key{$0$},最大值是\key{$2 \sqrt {5}$}.
\begin{solution}{4cm}

\end{solution}
