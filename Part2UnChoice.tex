\question[6] (6分)某同学利用图(a)所示装置验证动能定理。调整木板的倾角平衡摩擦阻力后,挂上钩码,钩码下落,带动小车运动并打出纸带。某次实验得到的纸带及相关数据如图(b)所示。已知打出图(b)中相邻两点的时间间隔为$0.02s$,从图(b)给出的数据中可以得到,打出B点时小车的速度大小$v_B=$\key{}$m/s﹐$打出Р点时小车的速度大小$v_P=$\key{}$m/s.$(结果均保留2位小数)若要验证动能定理,除了需测量钩码的质量和小车的质量外,还需要从图(b)给出的数据中求得的物理量为\key{}·
\question[6] (9分)已知一热敏电阻当温度从$10∘C$升至$60°C$时阻值从几千欧姆降至几百欧姆,某同学利用伏安法测量其阻值随温度的变化关系.所用器材:电源E、开关S、滑动变阻器R(最大阻值为$20Ω$)、电压表(可视为理想电表)和毫安表(内阻约为$100Ω$).答案

(1)答案在答题卡上所给的器材符号之间画出连线,组成测量电路图答案

(2)答案实验时,将热敏电阻置于温度控制室中,记录不同温度下电压表和毫安表的示数,计算出相应的热敏电阳阻值。若某次测量中电压表和毫安表的示数分别为$5.5V$和$3.0mA,$则此时热敏电阻的阻值为\key{}(保留2位有效数字)。实验中得到的该热敏电阻阻值R随温度t变化的曲线如图(a)所示答案

(3)答案将热敏电阻从温控室取出置于室温下,测得达到热平衡后热敏电阻的阻值为$2.2kΩ$。由图(a)求得,此时室温为\key{}◦C保留3位有效数字)。答案

(4)答案利用实验中的热敏电阻可以制作温控报警器,其电路的一部分如图(b)所示。图中,E为直流电源(电动势为$10V,$内阻可忽略);当图中的输出电压达到或超过$6.0V$时,便触发报警器(图中未画出)报警.若要求开始报警时环境温度为$50∘C$,则图中\key{}(填$"R_1"$或$"R_2"$)应使用热敏电阻,另一固定电阻的阻值应为\key{}kΩ(保留2位有效数字).
\question[6] (12分)如图,一边长为$l_0$的正方形金属框$abcd$固定在水平面内,空间存在方向垂直于水平面、磁感应强度大小为B的匀强磁场.一长度大于$\sqrt{2}l_{0}$的均匀导体棒以速率v自左向右在金属框上匀速滑过,滑动过程中导体棒始终与ac垂直且中点位于ac上,导体棒与金属框接触良好.已知导体棒单位长度的电阻为r,金属框电阻可忽略.将导体棒与a点之间的距离记为x,求导体棒所受安培力的大小随x($0\leqslant\sqrt{2}l_{0}$)变化的关系式.
\question[6] (20分)如图,相距$L=11.5m$的两平台位于同一水平面内,二者之间用传送带相接.传送带向右匀速运动,其速度的大小v可以由驱动系统根据需要设定.质量$m=10kg$的载物箱(可视为质点),以初速度$v_0=5.0m/s$自左侧平台滑上传送带.载物箱与传送带间的动摩擦因数$μ=0.10$,重力加速度取$g=10m/s^2.$答案

(1)答案若$v=4.0m/s$,求载物箱通过传送带所需的时间;答案

(2)答案求载物箱到达右侧平台时所能达到的最大速度和最小速度;答案

(3)答案若$v=6.0m/s$,载物箱滑上传送带$Δ\Delta t=\frac{13}{12}s$后,传送带速度突然变为零.求载物箱从左侧平台向右侧平台运动的过程中,传送带对它的冲量.
\question[6] [物理——选修$3–3$](15分)答案

(1)答案(5分)如图,一开口向上的导热气缸内。用活塞封闭了一定质量的理想气体,活塞与气缸壁间无摩擦。现用外力作用在活塞上。使其缓慢下降。环境温度保持不变,系统始终处于平衡状态。在活塞下降过程中\key{}。(填正确答案标号。选对1个得2分。选对2个得4分,选对3个得5分;每选错1个扣3分,最低得分为0分)A.气体体积逐渐减小,内能增知B.气体压强逐渐增大,内能不变C.气体压强逐渐增大,放出热量D.外界对气体做功,气体内能不变E.外界对气体做功,气体吸收热量答案

(2)答案(10分)如图,两侧粗细均匀、横截面积相等、高度均为$H=18cm$的U型管,左管上端封闭,右管上端开口.右管中有高$h_0=4cm$的水银柱,水银柱上表面离管口的距离$1=12cm.$管底水平段的体积可忽略.环境温度为$T_1=283K.$大气压强$P_0=76cmHg.$(i)现从右侧端口缓慢注入水银(与原水银柱之间无气隙),恰好使水银柱下端到达右管底部.此时水银柱的高度为多少?(ⅱ)再将左管中密封气体缓慢加热,使水银柱上表面恰与右管口平齐,此时密封气体的温度为多少?
\question[6] [物理选修$3–4$](15分)答案

(1)答案(5分)如图,一列简谐横波平行于x轴传播,图中的实线和虚线分别为$t=0$和$t=0.1s$时的波形图。已知平衡位置在$x=6m$处的质点,在0到$0.ls$时间内运动方向不变。这列简谐波的周期为\key{}s,波速为\key{}$m/s,$传播方向沿x轴\key{}(填“正方向”或“负方向”)。答案

(2)答案(10分)如图,一折射率为的材料制作的三棱镜,其横截面为直角三角形$ABC$,$∠A=90^∘$,$∠B=30^∘.$一束平行光平行于BC边从AB边射入棱镜,不计光线在棱镜内的多次反射,求AC边与BC边上有光出射区域的长度的比值.
